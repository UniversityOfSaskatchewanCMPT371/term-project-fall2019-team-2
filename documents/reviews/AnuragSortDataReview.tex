\documentclass{article}

\title{Code Review}
\author{Anurag - SortData Refactor}
\date{\today}

\begin{document}

\maketitle

\section{Roles}
Author: Anurag\\
Other Attendees: Braunson, Evan, Mesa\\

\section{Objectives}


\section{Scoring}
    Provide a rating from 1 - 5. \\
    1 - Work has room for improvement \\
    3 - Work was done to an acceptable standard \\
    5 - The work went above and beyond expectations\\

\begin{tabular}{|l|c|}
	\hline
	Criteria & Score\\
	\hline
    Completeness:
    (completed to the full scope of the requirements) & 4\\
	\hline
    Concise:
    (the purpose is achieved without undue complication) & 5\\
	\hline
    Documentation (Including author): & 4\\
    \hline
    Structure/organization:
    (does the work flow well, are components grouped logically?) & 5\\
    \hline
    Modularity: & 5\\
    \hline
    Follows code style: & 5\\
    \hline
\end{tabular}

\section{Notes}
    
What was done well? Documentation was done well, pre and post-conditions for everything new.  \\
All date string formats were tested. \\
    
Other notes: \\
    
\pagebreak

\section{List of Defects}

Lines 203-212 and 214-223 are basically doing the exact same thing, these should probably be abstracted into a singular function.\\ \\
The unit tests don't intermix various date formats, so we can't be sure that the system is robust if it finds different date formats in the same column \\ \\
Some negative testing is missing for the isValid method \\ \\
Look for date key can get to the end of the function without returning anything. This should probably be changed to throw an error if that happens. \\ \\
The checks if the data is sorted don't actually check that it's sorted, they just test that the data is different. \\ \\
Line 198 allows a key to be type 'any' this should be changed to string. \\ \\
Asserts should be used to make sure the key is never null or undefined. \\ \\
The result of lookForDateKey should be stored


\end{document}
